Per l'approccio con singola funzione per chiamata, anche in questo caso viene creata una \textit{Cloud Function} per ogni metodo HTTP da esporre.

Per il database vale il discorso del precedente approccio.

Gli script utilizzati tendono ad essere simili allo script della funzione unica, rimuovendo tutto il codice inerente alla gestione degli altri metodi HTTP.

Viene mostrato come riferimento solo uno dei quattro script, nello specifico il metodo \textit{PUT /items}:
\begin{lstlisting}[language=python]
import json
from google.cloud import firestore

client = firestore.Client()
collection_name = 'ProductInventory'

def function(request):
    request_json = request.get_json(silent=True)
    path = request.path
    method = request.method

    body = {}
    statusCode = 200

    try:
        if path == "/items" and method == "PUT":
            if not request_json:
                raise ValueError("No request body")
            item_id = request_json['id']
            doc_ref = client.collection(collection_name).document(item_id)
            doc_ref.set({
                'price': request_json['price'],
                'name': request_json['name'],
                'description': request_json['description']
            })
            body = f'Put item {item_id}'
        else:
            raise KeyError(f'Unsupported route: {method} {path}')
    except KeyError as e:
        statusCode = 400
        body = str(e)
    except Exception as e:
        statusCode = 500
        body = str(e)
    
    res = {
        "statusCode": statusCode,
        "headers": {
            "Content-Type": "application/json"
        },
        "body": json.dumps(body)
    }

    return (res["body"], res["statusCode"], res["headers"])
\end{lstlisting}
Ogni funzione avrà quindi una singola responsabilità: gestire un solo metodo HTTP.