\textit{DynamoDB} è un servizio di \textit{database cloud NoSQL} progettato per offrire prestazioni rapide e costanti a qualsiasi scala. Questo servizio, parte integrante dell'infrastruttura AWS, supporta centinaia di migliaia di clienti utilizzando una vasta rete di server distribuiti globalmente.

Gli utenti apprezzano DynamoDB per la sua capacità di rispondere alle richieste con una latenza costantemente bassa, con l'obiettivo di mantenere i tempi di risposta nell'ordine di pochi millisecondi.

DynamoDB si distingue per l'integrazione di alcune caratteristiche fondamentali, pensate per semplificare la vita di clienti e sviluppatori:
\begin{itemize}
    \item \textbf{Servizio completamente gestito}. Grazie all'API di DynamoDB, le applicazioni possono creare tabelle, leggere e scrivere dati senza doversi preoccupare della gestione fisica o del mantenimento del database. DynamoDB si occupa di tutto: provisioning delle risorse, recupero da guasti, crittografia dei dati, aggiornamenti software, backup e altre operazioni di gestione.
    \item \textbf{Architettura multi-tenant}. DynamoDB ospita i dati di diversi clienti sulle stesse risorse fisiche, ottimizzando l'uso delle risorse e permettendo un risparmio sui costi che viene trasferito ai clienti. Ogni carico di lavoro è isolato grazie al monitoraggio dell'utilizzo e al provisioning accurato delle risorse.
    \item \textbf{Scalabilità illimitata}. Non ci sono limiti predefiniti sulla quantità di dati che una tabella può contenere. DynamoDB si espande elasticamente in base alle esigenze delle applicazioni, scalando da pochi a migliaia di server per soddisfare la domanda di storage e throughput.
    \item \textbf{Prestazioni prevedibili}. L'API semplificata di DynamoDB consente di gestire le richieste con bassa latenza, che generalmente rimane nell'ordine di millisecondi per i dati archiviati nella stessa regione AWS. Le latenze restano stabili anche quando le tabelle crescono da pochi MB a centinaia di TB, grazie alla distribuzione dei dati e al bilanciamento del carico.
    \item \textbf{Alta disponibilità}. I dati sono replicati automaticamente su più \textit{Availability Zone} di AWS, garantendo continuità anche in caso di guasti a dischi o nodi, soddisfacendo così i requisiti più stringenti di disponibilità e durabilità.
    \item \textbf{Flessibilità d'uso}. DynamoDB non impone uno schema fisso per le tabelle, permettendo agli sviluppatori di creare modelli di dati personalizzati, inclusi attributi multivalore. Supporta modelli di dati chiave-valore o documento.
\end{itemize}
Queste caratteristiche fanno di DynamoDB una soluzione ideale per chi cerca un database cloud altamente scalabile, performante e affidabile.\cite{elhemali2022amazon}