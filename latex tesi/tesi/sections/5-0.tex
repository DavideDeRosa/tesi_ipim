Nel mondo delle funzioni serverless esistono molti \textit{pattern architetturali} utili per la scrittura di codice pulito, efficiente e sicuro. In questo capitolo verranno discussi due pattern per la scrittura di funzioni, ognuno con i propri punti di forza e debolezze.

I due approcci in questione possono essere definiti come:
\begin{itemize}
    \item Funzione unica per tutte le API
    \item Funzione per ogni chiamata API
\end{itemize}

Esempi dei due approcci saranno presenti nel \textit{caso studio}, dove saranno confrontati.