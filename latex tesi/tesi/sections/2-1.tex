Il \textit{Serverless Computing} è una tecnologia in rapida crescita che sta avendo un impatto sempre più significativo sulla società, trovando ampia adozione sia nel mondo accademico che in quello industriale. La sua promessa principale è rendere i servizi informatici più accessibili, personalizzabili in base alle esigenze specifiche e a basso costo, delegando all'infrastruttura la gestione dei problemi operativi.

I principali fornitori di servizi cloud, come \textit{Amazon}, \textit{Microsoft}, \textit{Google} e \textit{IBM}, offrono piattaforme serverless già pronte all'uso, con ben definite responsabilità e prezzi.

Un sistema può essere considerato serverless se presenta le seguenti caratteristiche\cite{7979855}:
\begin{itemize}
    \item \textbf{Auto-scaling}. La capacità di adattarsi automaticamente alle variazioni del carico di lavoro, scalando sia orizzontalmente che verticalmente, è un elemento chiave. Un'applicazione serverless può ridurre il numero di istanze fino a zero, introducendo il concetto di \textit{cold startup}, che può causare ritardi nel tempo di risposta dovuti alla necessità di avviare l'ambiente da zero e caricare il codice.
    \item \textbf{Pianificazione flessibile}. Non essendo vincolata a un server specifico, l'applicazione viene pianificata dinamicamente in base alle risorse disponibili nel cluster, garantendo bilanciamento del carico e prestazioni ottimali. Inoltre, la pianificazione tiene conto di più regioni geografiche per evitare interruzioni del servizio in caso di malfunzionamenti o crash.
    \item \textbf{Event-driven}. Le applicazioni serverless si attivano in risposta a eventi, come richieste HTTP, aggiornamenti di code di messaggi o nuove scritture su servizi di storage. Tramite l'associazione di trigger e regole agli eventi, il sistema è in grado di rispondere in modo efficiente e flessibile alle diverse tipologie di input. Gli eventi possono essere attivati non solo da fonti esterne alla piattaforma cloud, ma anche internamente, attraverso i vari servizi offerti dalla piattaforma stessa. Questo permette agli sviluppatori di creare applicazioni distribuite che utilizzano diversi servizi cloud in modo integrato. Il serverless computing rappresenta una parziale realizzazione di un modello basato sugli eventi, dove le applicazioni vengono definite dalle azioni e dagli eventi che le attivano. Questo concetto richiama i sistemi di database attivi e riflette la letteratura sui sistemi event-driven, che da tempo teorizza l'esistenza di sistemi informatici generali in cui le azioni sono elaborate in modo reattivo ai flussi di eventi.
    
    Le piattaforme serverless basate su funzioni adottano pienamente questa visione, utilizzando astrazioni semplici come le funzioni per definire le azioni e costruendo la logica di elaborazione degli eventi direttamente all'interno del cloud. In questo modo, il serverless computing offre un framework flessibile per la gestione e l’elaborazione degli eventi su larga scala.
    \item \textbf{Sviluppo trasparente}. Gli sviluppatori non devono più preoccuparsi della gestione delle risorse fisiche o dell'ambiente di esecuzione, poiché queste responsabilità sono delegate ai provider cloud. Questi ultimi si occupano di garantire la disponibilità delle risorse fisiche, la sicurezza e la potenza di calcolo, rendendo tutto ciò trasparente agli sviluppatori, facilitando così il processo di sviluppo e distribuzione.
    \item \textbf{Pagamento in base al consumo}. Il serverless trasforma il costo della capacità di calcolo da una spesa di capitale a una spesa operativa, eliminando la necessità per gli utenti di acquistare server dedicati per i picchi di carico. Il modello \textit{pay-as-you-go} permette di pagare solo per le risorse effettivamente utilizzate.
\end{itemize}

Un modello di calcolo che soddisfa queste cinque caratteristiche è considerato serverless. La sua crescente diffusione è dovuta in parte al fatto che gli sviluppatori possono pagare solo in base all'uso effettivo delle risorse, piuttosto che per una capacità preallocata.

Oggi, il serverless computing viene utilizzato principalmente in scenari backend per lavori batch, come l'analisi dei dati, attività di machine learning e applicazioni web basate su eventi.\cite{10.1145/3508360}
