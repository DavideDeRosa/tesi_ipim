Una volta creato e configurato il proprio account \textit{GCP} (\textit{Google Cloud Platform}), si dovrà creare o selezionare un \textit{progetto}, all'interno del quale verranno gestiti tutti i servizi utilizzati.

Selezionato il progetto, si può procedere con la ricerca dei servizi offerti dalla piattaforma attraverso la console di GCP che, simile ad AWS, funge da centro di controllo di tutti i servizi disponibili.

Cercando nei servizi offerti il servizio \textit{Funzioni Cloud Run} si accede alla console dedicata. E' possibile quindi creare nuove funzioni o selezionarne una creata in precedenza.

Per creare una nuova funzione, si deve selezionare l'\textit{ambiente} (la tipologia di \textit{cloud run function}), il \textit{nome della funzione}, la \textit{regione geografica} ed il \textit{tipo di trigger} (sono disponibili \textit{HTTPS}, \textit{Cloud Pub/Sub}, \textit{Cloud Storage}, \textit{Cloud Firestore}, \textit{altri trigger}). Selezionato il trigger, viene richiesto il \textit{tipo di autenticazione} alla funzione, potendo scegliere tra \textit{Consenti chiamate non autenticate} e \textit{Autenticazione necessaria}.\\
E' possibile inoltre configurare aspetti più tecnici, come impostazioni di \textit{runtime}, \textit{build}, \textit{connessioni} e \textit{repository per sicurezza e immagini}.

Configurato il tutto, un \textit{editor di codice} permette di scegliere il \textit{linguaggio di runtime} (sono disponibili diversi runtime per i linguaggi Python, NodeJS, Java, Ruby, .NET, Go e PHP) e di scegliere l'\textit{entry point} della cloud function (il nome della funzione da eseguire). Qui sarà possibile scrivere o incollare il codice della funzione, modificabile anche dopo la creazione.

Creata la funzione, si può accedere alla sua pagina dedicata. Vengono quindi mostrare le \textit{metriche} dettagliate della funzione, i \textit{dettagli}, il \textit{codice} (modificabile in qualsiasi momento), le diverse \textit{variabili} utili per \textit{runtime} e \textit{build}, i \textit{trigger} (come sarà possibile osservare a breve non c'è bisogno di alcun tipo di configurazione extra, a differenza di AWS), le \textit{autorizzazioni}, i \textit{log} ed i \textit{test}. 

\textit{Cloud Run}, a differenza di \textit{Lambda}, non necessità di ulteriori configurazioni per esporre API. Viene fornito direttamente un \textit{URL} dedicato alla funzione appena creata. Tutta la configurazione dell'instradamento dovrà essere effettuata all'interno del codice.
