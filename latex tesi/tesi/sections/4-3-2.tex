L'integrazione di Firestore con le Google Cloud Functions risulta essere molto lineare. Prendendo come esempio una funzione scritta in Python, basterà dichiarare nei \textit{requirements.txt} la libreria \textit{google-cloud-firestore}. Una volta dichiarata, all'interno dello script potrà essere utilizzato il modulo \textit{firestore}, il quale permette di stabilire una connessione con Firestore.

Specificando il nome della \textit{collection}, sarà possibile interagire con il nostro database Firestore. Non è necessaria alcuna configurazione di ruoli per accedere al database, a differenza di DynamoDB.

Alcuni esempi di codice verranno mostrati nel \textit{caso studio} successivamente.