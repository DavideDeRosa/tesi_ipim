Le grandi aziende tecnologiche come \textit{Amazon}, \textit{Google} e \textit{Microsoft} offrono piattaforme serverless sotto diversi marchi. Sebbene i dettagli dei servizi possano variare, il principio fondamentale è lo stesso: con il modello di calcolo a consumo, il serverless computing mira a garantire l'autoscaling e a offrire servizi di calcolo a costi contenuti.

Sono state introdotte diverse implementazioni commerciali di successo. Amazon ha lanciato \textit{Lambda} nel 2014, seguita da \textit{Google Cloud Functions}, \textit{Microsoft Azure Functions} e \textit{IBM OpenWhisk} nel 2016.\cite{shafiei2022serverless}

Tutte queste infrastrutture seguono il paradigma della \textit{programmazione funzionale}: una funzione rappresenta un'unità di software che può essere distribuita sull'infrastruttura cloud del provider ed eseguire un'unica operazione in risposta a un evento esterno. Le funzioni possono essere attivate da diversi tipi di eventi, come\cite{malawski2020serverless}:
\begin{itemize}
    \item un evento generato dall'infrastruttura cloud, ad esempio una modifica in un database cloud, il caricamento di un file in un object store, l'inserimento di un nuovo elemento in un sistema di messaggistica o un'azione programmata a un orario specifico;
    \item una richiesta diretta da parte dell'applicazione tramite HTTP o chiamate API del cloud.
\end{itemize}