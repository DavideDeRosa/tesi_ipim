I diversi test delle \textit{performance} hanno rivelato che \textit{AWS Lambda} offre una \textit{latenza inferiore} rispetto a \textit{Google Cloud Functions}, sia in condizioni di \textit{cold start} che in \textit{esecuzioni normali}. 

Inoltre, dal punto di vista economico, \textit{AWS Lambda} si è dimostrata \textbf{più conveniente}, con un \textbf{costo per richiesta inferiore} rispetto a \textit{Google}, sebbene Google offra un maggior numero di richieste gratuite mensili. 

Per quanto riguarda la facilità d'uso, entrambe le piattaforme presentano interfacce intuitive, ma \textit{Google Cloud Functions} si distingue per la semplificazione nel processo di deploy, eliminando la necessità di configurare ruoli aggiuntivi o Gateway API.

\newpage

La seguente tabella riassume i risultati ottenuti:
\begin{table}[htbp]
\centering
\begin{tabular}{lcc}
\toprule
 & AWS Lambda & Google Cloud Functions \\
\midrule
Performance & \textbf{+} & \textbf{-} \\
Costo per richiesta & \textbf{+} & \textbf{-} \\
Richieste gratuite & \textbf{-} & \textbf{+} \\
Facilità di deploy & \textbf{-} & \textbf{+} \\
Qualità documentazione & \textbf{+} & \textbf{-} \\
\bottomrule
\end{tabular}
\end{table}