Questo approccio rappresenta la forma più pura dei \textit{pattern serverless}. Ogni funzione ha un suo file, ed esegue una singola chiamata API.

I vantaggi di questo approccio sono:
\begin{itemize}
    \item Massima riusabilità del codice. Ogni funzione ha una singola responsabilità.
    \item Si viene spinti a scrivere codice maggiormente testabile.
    \item Minor carico cognitivo per gli sviluppatori che modificano la specifica funzione.
    \item Miglior ottimizzazione dei tempi di esecuzione, e di conseguenza dei costi.
\end{itemize}

Gli svantaggi invece sono:
\begin{itemize}
    \item Approccio funzionante solo per architetture completamente \textit{event-driven}.
    \item Soffermandosi sul quadro generale, il carico cognitivo aumenta quando si parla di modifiche a livello di sistema.
    \item La manutenzione aumenta man mano che le funzioni crescono di numero.
\end{itemize}