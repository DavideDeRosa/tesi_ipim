Come anticipato nel \textit{Capitolo 5}, il seguente approccio si basa su una singola funzione nella quale è presente tutto il codice per gestire le diverse chiamate API presenti.

Per quanto riguarda il database, viene utilizzato il precedentemente citato \textit{DynamoDB}. Per consentire il collegamento ai dati alla \textit{Lambda Function} occorre aggiungere alla funzione stessa il ruolo personalizzato discusso nel \textit{Capitolo 4}.

Viene quindi utilizzato il seguente script:
\begin{lstlisting}[language=python]
import json
import boto3

client = boto3.client('dynamodb')
dynamodb = boto3.resource("dynamodb")
table = dynamodb.Table('product-inventory')

def lambda_handler(event, context):
    body = {}
    statusCode = 200

    try:
        if event['routeKey'] == "DELETE /items/{id}":
            table.delete_item(Key={'id': event['pathParameters']['id']})
            body = 'Deleted item ' + event['pathParameters']['id']
        elif event['routeKey'] == "GET /items/{id}":
            body = table.get_item(Key={'id': event['pathParameters']['id']})
            body = body["Item"]
            responseBody = [{'price': body['price'], 'id': body['id'], 'name': body['name'], 'description': body['description']}]
            body = responseBody
        elif event['routeKey'] == "GET /items":
            body = table.scan()
            body = body["Items"]
            responseBody = []
            for items in body:
                responseItems = [{'price': items['price'], 'id': items['id'], 'name': items['name'], 'description': items['description']}]
                responseBody.append(responseItems)
            body = responseBody
        elif event['routeKey'] == "PUT /items":
            requestJSON = json.loads(event['body'])
            table.put_item(
                Item={
                    'id': requestJSON['id'],
                    'price': requestJSON['price'],
                    'name': requestJSON['name'],
                    'description': requestJSON['description']
                })
            body = 'Put item ' + requestJSON['id']
    except KeyError:
        statusCode = 400
        body = 'Unsupported route: ' + event['routeKey']
    
    body = json.dumps(body)
    res = {
        "statusCode": statusCode,
        "headers": {
            "Content-Type": "application/json"
        },
        "body": body
    }

    return res
\end{lstlisting}

Lo script è strutturato in maniera lineare. Dopo aver stabilito la connessione con il database (grazie alla libreria \textit{boto3}) si definisce l'entry-point della \textit{Lambda Function}.

Il cuore della funzione controlla a quale metodo HTTP appartiene la richiesta e lo gestisce di conseguenza, restituendo il risultato dell'operazione. In caso di richiesta errata, viene gestito l'errore restituendo un messaggio di errore.

E' possibile quindi osservare l'approccio a funzione unica, con tutta la logica presente all'interno di una singola \textit{Lambda}. I diversi percorsi creati in \textit{API Gateway} puntano alla stessa funzione, che gestisce tutti i metodi HTTP.