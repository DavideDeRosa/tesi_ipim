Questa tesi ha esaminato due delle principali piattaforme per l’esecuzione di funzioni serverless oggi disponibili, \textit{AWS Lambda} e \textit{Google Cloud Functions}, con l’obiettivo di confrontarne le \textit{prestazioni}, i \textit{costi} e la \textit{facilità di deploy}. 

Attraverso un caso studio che prevedeva l'integrazione con database serverless come \textit{Amazon DynamoDB} e \textit{Google Firestore}, sono stati messi alla prova due approcci architetturali differenti: la \textit{funzione unica per tutte le API} e la \textit{funzione dedicata per ciascuna chiamata API}.