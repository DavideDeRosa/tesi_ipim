Per le \textit{performance} si è deciso di testare la \textit{latenza} dell'API.

Per \textit{latenza} si intende il tempo che un'API impiega per elaborare una richiesta ed inviare una risposta, compresi eventuali ritardi di rete o di elaborazione. Può essere influenzata da vari fattori, come la velocità della connessione di rete, il tempo di elaborazione dell'API e la quantità di dati trasferiti.

Nel nostro caso andremo a testare la latenza ottenuta in caso di \textit{cold startup} e non, osservando le possibili differenze anche tra i due approcci di API.

La fase di testing consiste nel misurare la latenza effettuando una media di 10 chiamate, escludendo i due valori estremi.\\
In entrambi i casi viene testato il metodo \textit{GET /items}.

I risultati ottenuti per \textit{Lambda} sono:
\begin{table}[H]
    \centering
    \begin{tabular}{|c|c|c|}
    \hline
        \textbf{} & \textbf{Funzione Unica} & \textbf{Funzione Singola} \\ \hline
        Cold Startup & 585 ms & 556 ms \\ \hline
        Normale & 84 ms & 86 ms \\ \hline
    \end{tabular}
\end{table}

I risultati ottenuti per \textit{Cloud Functions} sono:
\begin{table}[H]
    \centering
    \begin{tabular}{|c|c|c|}
    \hline
        \textbf{} & \textbf{Funzione Unica} & \textbf{Funzione Singola} \\ \hline
        Cold Startup & 857 ms & 802 ms \\ \hline
        Normale & 113 ms & 106 ms \\ \hline
    \end{tabular}
\end{table}