In una piattaforma serverless, l'utente si limita a scrivere una funzione cloud in un \textit{linguaggio di alto livello} e a specificare l'evento che ne deve innescare l'esecuzione, ad esempio il caricamento di un'immagine nello storage cloud o l'inserimento di una miniatura in una tabella del database. Il sistema serverless si occupa poi di gestire tutto il resto, inclusi la selezione dell'istanza, la scalabilità, la distribuzione, la tolleranza ai guasti, il monitoraggio, la registrazione e l'applicazione di patch di sicurezza. 

L'approccio tradizionale, che può essere definito anche \textit{serverfull computing}, può essere visto come la programmazione in un linguaggio assembly di basso livello, mentre il serverless computing assomiglia alla programmazione in un linguaggio di alto livello, come Python. Un programmatore che utilizza un linguaggio assembly per calcolare un'espressione semplice come \textit{c = a + b} deve scegliere i registri, caricare i valori nei registri, eseguire l'operazione aritmetica e infine memorizzare il risultato. Questo processo riflette molte delle fasi del serverful computing nel cloud: prima si allocano o identificano le risorse, poi si caricano con il codice e i dati necessari, si eseguono i calcoli, si memorizzano i risultati e infine si rilasciano le risorse. L'obiettivo del serverless computing è quello di semplificare questo processo, offrendo ai programmatori cloud vantaggi simili a quelli della programmazione in linguaggi di alto livello.

Altre caratteristiche degli ambienti di programmazione avanzati trovano un parallelo naturale nel serverless computing. Ad esempio, la gestione automatica della memoria solleva i programmatori dal gestire le risorse di memoria; allo stesso modo, il serverless computing libera i programmatori dal dover gestire direttamente le risorse del server.

In particolare, ci sono tre differenze fondamentali tra il serverless computing e quello serverful:
\begin{itemize}
    \item \textbf{Calcolo e storage disaccoppiati}. Nel serverless computing, lo storage e la computazione scalano in modo indipendente e vengono forniti e tariffati separatamente. In genere, lo storage è gestito tramite un servizio cloud dedicato, mentre la computazione avviene in modo stateless.
    \item \textbf{Esecuzione del codice senza gestione delle risorse}. L'utente non deve più preoccuparsi di allocare risorse. Basta fornire il codice, e il cloud si occupa automaticamente di assegnare le risorse necessarie per l'esecuzione.
    \item \textbf{Pagare per le risorse effettivamente utilizzate}. La fatturazione avviene in base alle risorse effettivamente consumate, come il tempo di esecuzione, anziché alle risorse preallocate, come la dimensione e il numero di macchine virtuali.
\end{itemize}

Usando queste differenze, si può spiegare come l'approccio serverless si distingue da soluzioni simili, presenti e passate.

L'attuale serverless computing con funzioni cloud si differenzia dai suoi predecessori per diversi aspetti essenziali: migliore autoscaling, forte isolamento, flessibilità della piattaforma e supporto dell'ecosistema dei servizi. Tra questi fattori, l'autoscaling offerto da \textit{AWS Lambda} ha segnato un netto distacco da quanto fatto in precedenza. Ha seguito il carico con una fedeltà molto maggiore rispetto alle tecniche di autoscaling basate sui server, rispondendo rapidamente per scalare quando necessario e scendendo fino a zero risorse e zero costi in assenza di domanda. La tariffazione è molto più precisa, con un incremento minimo di fatturazione di \textit{100 ms} in un periodo in cui altri servizi di autoscaling prevedono la tariffazione oraria.

In particolare, il cliente viene addebitato per il tempo in cui il codice \textit{viene effettivamente eseguito}, non per le risorse riservate all'esecuzione del programma. Questa distinzione ha garantito che il cloud provider fosse “coinvolto nel gioco” dell'autoscaling e di conseguenza ha fornito incentivi per garantire un'allocazione efficiente delle risorse.

Per i cloud provider, il serverless computing favorisce la crescita del business, in quanto rendere il cloud più facile da programmare aiuta ad attirare nuovi clienti e a far sì che i clienti esistenti utilizzino maggiormente le offerte cloud.

Il breve tempo di esecuzione, il ridotto utilizzo della memoria e la natura stateless migliorano il multiplexing statistico, facilitando ai fornitori di servizi cloud la ricerca di risorse inutilizzate per eseguire questi compiti. I fornitori di cloud possono anche sfruttare hardware meno recente, come server più vecchi che potrebbero risultare meno interessanti per i clienti dei servizi di cloud tradizionali. Questi vantaggi contribuiscono ad aumentare il reddito derivante dalle risorse esistenti.

I clienti, dal canto loro, traggono vantaggio da una maggiore produttività nella programmazione e, in molti casi, possono anche ottenere risparmi sui costi grazie al miglior utilizzo delle risorse sottostanti. Anche se il serverless computing consente una maggiore efficienza l'uso del cloud potrebbe aumentare piuttosto che diminuire, poiché l'efficienza superiore stimola una maggiore domanda e l'ingresso di nuovi utenti.

Inoltre, il serverless computing eleva il livello di astrazione del cloud dal codice macchina \textit{x86} (che rappresenta il 99\% dei computer cloud) ai linguaggi di programmazione di alto livello, permettendo così innovazioni architettoniche. Se architetture come \textit{ARM} o \textit{RISC-V} offrono migliori prestazioni in termini di costi rispetto a \textit{x86}, il serverless computing facilita il passaggio a nuovi set di istruzioni. I fornitori di cloud potrebbero persino adottare ricerche su ottimizzazioni basate sui linguaggi e su architetture specifiche per domini, al fine di accelerare l'esecuzione di programmi scritti in linguaggi come Python.