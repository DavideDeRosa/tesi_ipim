Il \textit{caso studio} viene accompagnato da diversi script \textit{Python}, leggermente diversi tra le due piattaforme e tra i due approcci.

L'obiettivo della soluzione software proposta è quella di fornire funzionalità \textit{CRUD} ipotizzando la presenza di un \textit{inventario} da riempire con dei \textit{prodotti}. Tutti i dati inerenti all'inventario verranno quindi salvati in un database.

La soluzione presenta i seguenti metodi HTTP:
\begin{itemize}
    \item \textbf{GET /items}: permette di ottenere la lista di tutti i prodotti presenti nell'inventario.
    \item \textbf{PUT /items}: permette di inserire o modificare un prodotto presente nell'inventario.
    \item \textbf{GET /items/\{id\}}: permette di ottenere un singolo prodotto dell'inventario specificando un \textit{identificativo} univoco, se presente.
    \item \textbf{DELETE /items/\{id\}}: permette di eliminare un singolo prodotto dell'inventario specificando un \textit{identificativo} univoco.
\end{itemize}