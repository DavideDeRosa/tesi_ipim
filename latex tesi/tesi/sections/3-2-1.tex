Google ha lanciato \textit{Google Cloud Functions} in modo piuttosto discreto nel febbraio 2016. Pensata principalmente per i servizi di \textit{Google Cloud}, Google evidenzia diversi casi d'uso specifici per Google Cloud Functions, come \textit{backend per applicazioni mobili}, sviluppo di \textit{API} e \textit{microservizi}, \textit{elaborazione dati}, \textit{webhook} (per rispondere a \textit{trigger} di terze parti) e \textit{applicazioni IoT}.\cite{lynn2017preliminary}

Come per le \textit{Lambda Function}, viene eliminata la necessità per gli utenti di gestire direttamente i server, delegando la gestione dell'infrastruttura sottostante alla piattaforma.