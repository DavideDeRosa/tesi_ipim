La crescente diffusione del \textbf{Serverless Computing} ha rivoluzionato il modo in cui vengono sviluppate e distribuite le applicazioni Cloud, offrendo scalabilità automatica e gestione trasparente delle risorse. Questa tesi si propone di analizzare due delle principali piattaforme serverless attualmente disponibili: \textit{AWS Lambda} e \textit{Google Cloud Functions}, confrontandole in termini di performance, costi e facilità di deploy.

Attraverso un caso studio sono state implementate API utilizzando due approcci architetturali differenti (\textit{funzione unica per tutte le API} e \textit{funzione per ogni chiamata API}). Le API sono state successivamente integrate con i database serverless \textit{Amazon DynamoDB} e \textit{Google Firestore}. 

I risultati hanno dimostrato che \textit{AWS Lambda} offre tempi di risposta inferiori, specialmente durante il \textit{cold start}, e un costo per richiesta più vantaggioso. Tuttavia, \textit{Google Cloud Functions} si distingue per la sua semplicità di configurazione, riducendo il numero di passaggi necessari per il deploy.

La tesi conclude che, sebbene \textit{AWS Lambda} sia la scelta preferibile per progetti ad alte prestazioni e su larga scala, \textit{Google Cloud Functions} rappresenta un'opzione decisamente valida per piccoli team o progetti che necessitano di una configurazione rapida e intuitiva. 

Gli sviluppi futuri potrebbero includere l'analisi di altre piattaforme e l'applicazione del modello serverless in scenari più complessi, quali il \textit{machine learning} e l'\textit{intelligenza artificiale}.