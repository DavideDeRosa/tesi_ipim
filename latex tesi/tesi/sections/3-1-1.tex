Uno dei primi servizi di serverless computing è \textit{AWS Lambda}, che permette di eseguire funzioni stateless scritte in uno dei linguaggi di programmazione supportati (come Node.js, Java, C\# e Python) in risposta a eventi, su larga scala, con la possibilità di gestire fino a \textit{3000 invocazioni in parallelo}. 

Diversamente dai tradizionali servizi Cloud \textit{IaaS}, AWS Lambda elimina la necessità per gli utenti di gestire direttamente i server, offrendo un'elasticità automatica gestita dalla piattaforma. Le funzioni Lambda sono progettate per essere \textit{stateless}, ovvero non dipendono dall'infrastruttura sottostante. Il notevole livello di parallelismo supportato è una delle caratteristiche distintive di AWS Lambda.\cite{gimenez2019framework}