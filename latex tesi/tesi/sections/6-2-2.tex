L'approccio a funzione per ogni chiamata si distingue dal precedente per la sua suddivisione in diverse \textit{Lambda Function}, ognuna delegata alla gestione di un singolo metodo HTTP.

Per il database vale il discorso del precedente approccio.

Gli script utilizzati sono simili allo script della funzione unica, rimuovendo la presenza degli altri metodi HTTP non gestiti dalla specifica funzione.

Viene mostrato come riferimento solo uno dei quattro script, nello specifico il metodo \textit{GET /items/\{id\}}:
\begin{lstlisting}[language=python]
import json
import boto3

client = boto3.client('dynamodb')
dynamodb = boto3.resource("dynamodb")
table = dynamodb.Table('product-inventory')

def lambda_handler(event, context):
    body = {}
    statusCode = 200

    try:
        body = table.get_item(Key={'id': event['pathParameters']['id']})
        body = body["Item"]
        responseBody = [{'price': body['price'], 'id': body['id'], 'name': body['name'], 'description': body['description']}]
        body = responseBody
    except KeyError:
        statusCode = 400
        body = 'Unsupported route: ' + event['routeKey']
    
    body = json.dumps(body)
    res = {
        "statusCode": statusCode,
        "headers": {
            "Content-Type": "application/json"
        },
        "body": body
    }

    return res
\end{lstlisting}

E' quindi possibile osservare come la struttura sia la medesima tra i due approcci, dividendo però la funzione unica in quattro funzioni specifiche, ognuna con una responsabilità differente.

In questo caso, i percorsi creati in \textit{API Gateway} puntano alla funzione specifica.