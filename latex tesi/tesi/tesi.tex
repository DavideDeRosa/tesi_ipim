\documentclass[12pt,a4paper,twoside]{book}

\usepackage[utf8]{inputenc}
\usepackage[a4paper,inner=3.5cm,outer=2.5cm]{geometry}

\usepackage[titletoc,title,toc,page]{appendix}
\usepackage{verbatim}
\usepackage{placeins}
\usepackage{listings}
\usepackage{hyperref}
\usepackage[italian]{babel}
\usepackage{tikz}
\usepackage{parskip}

\usepackage{graphicx}
\usepackage{blindtext}
\usepackage{chngcntr}
\counterwithin{table}{chapter}

\usepackage{newlfont}
\usepackage{fancyhdr}
\usepackage{indentfirst}
\usepackage[utf8]{inputenc}
\usepackage{float}
\usepackage{hyperref}
\usepackage[capitalize,noabbrev]{cleveref}
\usepackage{soul}
\usepackage[font=footnotesize,labelfont=bf]{caption}

\usepackage{multirow}
\usepackage{hyphenat}
\hyphenation{mate-mati-ca recu-perare}

\usepackage{lscape} 

\usepackage{natbib}
\bibliographystyle{alpha}
\setcitestyle{super,open={[},close={]}}

\newcommand{\rom}[1]{\uppercase\expandafter{\romannumeral #1\relax}}

\usepackage{pdfpages}

\begin{document}
% Per spostare i vari elementi più su o più giù gioca con i valori di vspace che ci sono tra uno e l'altro
\pagestyle{empty}
\newgeometry{left=2cm, right=2cm}
\begin{titlepage}
\begin{center}
    {{\Large{\textsc{Alma Mater Studiorum $\cdot$ Università di Bologna}}}}
    \rule[0.1cm]{\textwidth}{0.1mm}
    \rule[0.5cm]{\textwidth}{0.6mm}\\
    {\small{\bf SCUOLA DI SCIENZE\\
    Corso di Laurea in Informatica per il Management}}
\end{center}

\vspace{25mm}

\begin{center}
    {\LARGE{\bf ANALISI COMPARATIVA }}\\
    \vspace{3mm}
    {\LARGE{\bf DI SOLUZIONI SERVERLESS}}\\
\end{center}

\vspace{60mm}
\par
\noindent
\begin{minipage}[t]{0.04\textwidth}
~
\end{minipage}
\begin{minipage}[t]{0.4\textwidth}
{\large{\bf Relatore:\\
Chiar.mo Prof.\\
Rossi Davide}}
\end{minipage}
\hfill
\begin{minipage}[t]{0.4\textwidth}\raggedleft
{\large{\bf Presentata da:\\
De Rosa Davide}}
\end{minipage}
\begin{minipage}[t]{0.04\textwidth}
~
\end{minipage}

\vspace{30mm}

\begin{center}
    {\large{\bf \rom{2} Sessione\\
    Anno Accademico 2023/2024 }}
\end{center}
\end{titlepage}

\restoregeometry
\newpage
\begin{center}
    (DA FARE ALLA FINE)\\
    5 parole chiave per caratterizzare il contenuto della dissertazione:\\ (se non ti piacciono così sparse puoi anche semplicamente scriverle su una riga sola)
\end{center}

% https://tex.stackexchange.com/questions/26538/words-scattered-randomly-in-on-coverpage
\begin{tikzpicture}[overlay,remember picture,shift=(current page.center)]
\pgfmathsetseed{3}


\foreach [count=\count] \word in {Parola 1, parola 2, parola 3, parola 4, parola 5} {
\node [
    xshift={(mod(\count,3)-1)*(\paperwidth/4)},
    yshift={(mod(\count,7)-3)*(\paperwidth/6)},
    xshift=rand*4cm,
    yshift=rand*2cm,
    % rotate=rand*35,
    % opacity=rnd*0.5+0.125,
    font=\large] {\word};
}
\end{tikzpicture}
\newpage

\newpage~\newpage
\pagenumbering{gobble}
\chapter*{Abstract}
abstract

\topmargin=-1cm
\tableofcontents
\thispagestyle{empty}
\listoftables
\thispagestyle{empty}
\listoffigures
\thispagestyle{empty}
\newpage~\newpage

\pagenumbering{arabic}
\setcounter{chapter}{0}
\raggedbottom
\chapter{Introduzione} \label{chap:intro}
\pagestyle{plain}
\setcounter{page}{1}
da fare alla fine
\section{Scopo della Tesi}
Introduzione agli obiettivi della tesi, come il confronto tra soluzioni serverless e l'analisi di AWS e Google Cloud.

\section{Metodologia}
Descrizione dell'approccio adottato per l'analisi e il confronto delle due piattaforme.
 
\section{Struttura della Tesi}
Breve descrizione dei capitoli successivi.

\chapter{Nozioni di base su Serverless}

\section{Definizione e Concetti Fondamentali}
Introduzione al concetto di computing serverless, spiegando cosa si intende per "serverless" e quali sono i suoi principi di base (e.g., scalabilità automatica, pricing per utilizzo).

\section{Funzioni Serverless}
Spiegazione di cosa sono le funzioni serverless (FaaS - Function as a Service), come funzionano, e quali sono i loro principali vantaggi e svantaggi rispetto all'approccio tradizionale.

\chapter{Introduzione ad AWS Lambda e Google Cloud Functions}

\section{AWS Lambda}
\subsection{Panoramica di AWS Lambda}
Breve storia e introduzione di AWS Lambda.

\subsection{Caratteristiche Principali}
Descrizione delle caratteristiche principali di AWS Lambda (e.g., trigger, runtime supportati, integrazioni).

\subsection{Deploy su AWS Lambda}
Descrizione del processo di deploy di funzioni serverless su AWS.

\section{Google Cloud Functions}
\subsection{Panoramica di Google Cloud Functions}
Breve introduzione a Google Cloud Functions.

\subsection{Caratteristiche Principali}
Descrizione delle caratteristiche principali di Google Cloud Functions.

\subsection{Deploy su Google Cloud Functions}
Spiegazione del processo di deploy su Google Cloud.

\chapter{Integrazione con Database NoSQL}

\section{Introduzione ai Database NoSQL}
\subsection{Caratteristiche dei Database NoSQL}
Panoramica sui database NoSQL, con un focus su scalabilità, flessibilità del modello di dati e performance.

\subsection{Vantaggi dell'Utilizzo di NoSQL in un Contesto Serverless}
Spiegazione di come i database NoSQL siano particolarmente adatti per architetture serverless.

\section{Amazon DynamoDB}
\subsection{Panoramica su DynamoDB}
Introduzione a DynamoDB, il database NoSQL di AWS.

\subsection{Integrazione di AWS Lambda con DynamoDB}
Spiegazione di come le funzioni AWS Lambda interagiscono con DynamoDB, incluso l'utilizzo di trigger, accessi e operazioni CRUD (Create, Read, Update, Delete).

\section{Google Cloud Firestore}
\subsection{Panoramica su Firestore}
Introduzione a Google Cloud Firestore, il database NoSQL di Google Cloud.

\subsection{Integrazione di Google Cloud Functions con Firestore}
Spiegazione di come le funzioni Google Cloud interagiscono con Firestore, includendo l'accesso, le operazioni CRUD, e l'utilizzo di trigger.

\chapter{Architettura delle API Serverless}
\section{Approccio 1: Funzione Unica per API}
\subsection{Descrizione dell’Approccio}
Descrizione dell’Approccio

\subsection{Implementazione su AWS Lambda}
Implementazione su AWS Lambda

\subsection{Implementazione su Google Cloud Functions}
Implementazione su Google Cloud Functions

\subsection{Vantaggi e Svantaggi}
Vantaggi e Svantaggi

\section{Approccio 2: Funzione per Ogni Chiamata API}
\subsection{Descrizione dell’Approccio}
Descrizione dell’Approccio

\subsection{Implementazione su AWS Lambda}
Implementazione su AWS Lambda

\subsection{Implementazione su Google Cloud Functions}
Implementazione su Google Cloud Functions

\subsection{Vantaggi e Svantaggi}
Vantaggi e Svantaggi

\chapter{Analisi Comparativa tra AWS Lambda e Google Cloud Functions}
\section{Performance}
Tempo di Esecuzione e Latency
altro?

\section{Costi}
costi, non credo ci sia bisogno di distizione tra i due approcci, il numero di chiamate dovrebbe essere lo stesso

\section{Integrazioni e Compatibilità}
magari anche facilità di collegamento tra i diversi servizi (function e db)

\chapter{Caso Studio: Confronto tra le Due Soluzioni}
\section{Descrizione delle Soluzioni Software}
descrizione

\section{Implementazione su AWS Lambda}
\subsection{Deploy dell’Approccio Funzione Unica}
Deploy dell’Approccio Funzione Unica

\subsection{Deploy dell’Approccio Funzione per Ogni Chiamata}
Deploy dell’Approccio Funzione per Ogni Chiamata

\subsection{Risultati e Analisi}
Risultati e Analisi

\section{Implementazione su Google Cloud Functions}
\subsection{Deploy dell’Approccio Funzione Unica}
Deploy dell’Approccio Funzione Unica

\subsection{Deploy dell’Approccio Funzione per Ogni Chiamata}
Deploy dell’Approccio Funzione per Ogni Chiamata

\subsection{Risultati e Analisi}
Risultati e Analisi

\section{Confronto dei Risultati}
\subsection{Performance e Scalabilità}
Performance e Scalabilità

\subsection{Costi e Efficienza}
Costi e Efficienza

\subsection{Usabilità e Facilità di Deploy}
Usabilità e Facilità di Deploy

\chapter{Discussione dei Risultati}
considerazioni generali, limiti dello studio o altro

\chapter{Conclusioni}
sintesi dei risultati e conclusioni

\renewcommand{\bibsection}{}
\chapter*{Riferimenti bibliografici}
\bibliography{refs}
\end{document}